\section{Earth System Data Middleware}%
\label{earth-system-data-middleware}

The middleware for earth system data is a prototype to improve I/O
performance for earth system simulation as used in climate and weather
applications. ESDM exploits structural information exposed by workflows,
applications as well as data description formats such as HDF5 and NetCDF
to more efficiently organize metadata and data across a variety of
storage backends.

%\href{https://travis-ci.org/ESiWACE/esdm}{\includegraphics{https://travis-ci.org/ESiWACE/esdm.svg?branch=master}}

\subsection{Requirements}%
\label{requirements}

A compiler for C99 such as GCC5.0

\subsection{Installation}%
\label{installation}

Ensure you cloned the repository with the required submodules:

\begin{itemize}
  \item git clone -\/-recurse-submodules Or to initialize the submodules after the cloning:
  \item git submodule update -\/-init -\/-recursive
\end{itemize}

The installation instructions for the full stack with ESDM and NetCDF
can be found in the Dockerfiles that we use for testing.

\subsubsection{Ubuntu 20.04}%
\label{ubuntu-2004}

\$ grep RUN dev/docker/ubuntu-whole-stack/Dockerfile

\paragraph{Installation for CentOS7}%
\label{installation-for-centos7}

\$ grep RUN dev/docker/centos7-whole-stack/Dockerfile

\paragraph{Installation for FedoraSystem}%
\label{installation-for-fedora-system}

\begin{itemize}
  \item dnf install glib2 glib2-devel mpi jansson jansson-devel
  \item dnf install mpich-3.0 mpich-3.0-devel OR dnf install openmpi
    opemmpi-devel
  \item dnf install gcc-c++ gcc libtool cmake
\end{itemize}

\paragraph{Installation with Spack}%
\label{installation-with-spack}

Installation of spack itself:

\begin{itemize}
  \item git clone \url{https://github.com/spack/spack}
  \item export PATH=\$PATH:\$PWD/spack/bin/ Check that a suitable compiler is
    found
  \item spack compilers
\end{itemize}

First get the recent GCC:

\begin{itemize}
  \item spack install \href{mailto:gcc@9.3.0}{\nolinkurl{gcc@9.3.0}} Then
    install the packages with GCC:
  \item spack install jansson\%\href{mailto:gcc@9.3.0}{\nolinkurl{gcc@9.3.0}}
    glib\%\href{mailto:gcc@9.3.0}{\nolinkurl{gcc@9.3.0}}
    openmpi\%\href{mailto:gcc@9.3.0}{\nolinkurl{gcc@9.3.0}}
    gettext\%\href{mailto:gcc@9.3.0}{\nolinkurl{gcc@9.3.0}} Before running
    configure load the modules:
  \item spack load jansson\%\href{mailto:gcc@9.3.0}{\nolinkurl{gcc@9.3.0}}
    glib\%\href{mailto:gcc@9.3.0}{\nolinkurl{gcc@9.3.0}}
    openmpi\%\href{mailto:gcc@9.3.0}{\nolinkurl{gcc@9.3.0}} gcc
    gettext\%\href{mailto:gcc@9.3.0}{\nolinkurl{gcc@9.3.0}} -r
\end{itemize}

\subsection{Development}%
\label{development}

\subsubsection{Project directory structure}%
\label{project-directory-structure}

\begin{itemize}
  \item \texttt{dev} contains helpers for development purposes. For example,
    this project requires a development variant of HDF5 that provides the
    Virtual Object Layer (VOL). This and other dependencies can be
    installed into a development environment using the following script:

    \begin{lstlisting}
    ./dev/setup-development-environment.sh
    \end{lstlisting}
  \item \texttt{src} contains the source code... To build the project call:

    \begin{lstlisting}
    source dev/activate-development-environment.bash
    ./configure --debug
    cd build
    make -j
    \end{lstlisting}

    To run the test suite call:

    \begin{lstlisting}
    cd build
    make test
    \end{lstlisting}

    You may also choose to configure with a different hdf5 installation
    (see ./configure -\/-help) e.g.:

    \begin{lstlisting}
    ./configure --with-hdf5=$PWD/install
    \end{lstlisting}
  \item \texttt{doc} contains documentation which uses doxygen to build a HTML
    or a PDF version:

    For the HTML Reference use the following commands (assuming
    ./configure completed successfully):

    \begin{lstlisting}
    cd build
    doxygen
    build/doc/html/index.html
    \end{lstlisting}

    For a PDF Reference (requries LaTeX) run:

    \begin{lstlisting}
    cd build
    doxygen
    cd /doc/latex
    make
    \end{lstlisting}
  \item \texttt{tools} contains separate programs, e.g., for benchmarking
    HDF5. They should only be loosely coupled with the source code and
    allow to be used with the regular HDF5.
\end{itemize}

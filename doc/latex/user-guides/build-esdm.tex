\section{Building instructions}%
\label{installation-instructions-for-mistral}
%Mistral (HLRE-3) is the supercomputer installed at DKRZ in 2015/2016.
This guide documents installation procedures to build prerequisites as well as the prototype code base for development and testing purposes.

\begin{figure}[!ht]
  \begin{center}
    \includegraphics[width=0.6\textwidth]{../assets/deps/graph.png}
  \end{center}
  \caption{Dependencies overview}
  \label{fig:deps}
\end{figure}

\subsection{Dependencies (Spack)}%
\label{satisfying-requirements}
Although, the software trees on supercomputer may provide all the standard tools and libraries required to build ESDM, it is still recommended to build, to install, and to manage dependencies via Spack\footnote{\url{https://spack.readthedocs.io/en/latest/index.html}}.
The instructions below show how to setup Spack and how to install the ESDM dependencies.

\begin{enumerate}
  \item Download and enable Spack
    \begin{lstlisting}
    git clone --depth=2 https://github.com/spack/spack.git spack
    . spack/share/spack/setup-env.sh
    \end{lstlisting}
  \item Set a \lstinline|gcc| version to be used
    \begin{lstlisting}
    gccver=7.3.0
    \end{lstlisting}
  \item Install dependencies
    \begin{lstlisting}
    spack install gcc@$gccver +binutils
    spack compiler find
    spack install autoconf%gcc@$gccver
    spack install openmpi%gcc@$gccver gettext%gcc@$gccver
    spack install jansson%gcc@$gccver glib%gcc@$gccver
    \end{lstlisting}
  \item Load dependencies
    \begin{lstlisting}
    spack load gcc@$gccver
    spack load -r autoconf%gcc@$gccver
    spack load -r libtool%gcc@$gccver
    spack load -r openmpi%gcc@$gccver
    spack load -r jansson%gcc@$gccver
    spack load -r glib%gcc@$gccver
    \end{lstlisting}
\end{enumerate}


\subsection{ESDM prototype}
Assuming all prerequisites have been installed and tested, ESDM can be configured and build as follows.
\begin{enumerate}
  \item Ensure environment is aware of dependencies installed using \lstinline|spack| and \lstinline|dev-env|
    \begin{lstlisting}
    git clone https://github.com/ESiWACE/esdm
    \end{lstlisting}
  \item Configure and build ESDM
    \begin{lstlisting}
    cd esdm
    pushd deps
    ./prepare.sh
    popd
    ./configure \
      --prefix=${PREFIX} \
      --enable-netcdf4
    cd build
    make
    make install
    \end{lstlisting}
\end{enumerate}

\subsection{HDF5}%
\label{hdf5}
Assuming all prerequisites have been installed and tested, a development branch of HDF5-VOL can be build as follows.
\begin{enumerate}
  \item Ensure environment is aware of dependencies installed using \lstinline|spack| and \lstinline|dev-env|
  \item Checkout HDF5 with VOL support
    \begin{lstlisting}
    svn checkout https://svn.hdfgroup.org/hdf5/features/vol/
    \end{lstlisting}
  \item Configure and build HDF5-VOL
    \begin{lstlisting}
    cd vol
    ./autogen.sh
    export CC=mpicc         ### parallel features require mpicc
    ../configure \
      --prefix=$prefix \
      --enable-parallel \
      --enable-build-mode=debug \
      --enable-hl \
      --enable-shared
    make -j
    make install
    \end{lstlisting}
\end{enumerate}

\subsection{ESDM-NetCDF and ESDM-NetCDF-Python}%
\begin{enumerate}
  \item Clone the NetCDF-ESDM repository
    \begin{lstlisting}
    git clone https://github.com/ESiWACE/esdm-netcdf-c
    \end{lstlisting}
  \item Configure and build NetCDF-ESDM. (\lstinline|$INSTPATH| is the installation path of ESDM and HDF5-VOL.)
    \begin{lstlisting}
    cd esdm-netcdf-c
    /bootstrap
    ./configure \
      --prefix=$prefix \
      --with-esdm=$INSTPATH \
      LDFLAGS="-L$INSTPATH/lib" \
      CFLAGS="-I$INSTPATH/include" \
      CC=mpicc \
      --disable-dap
    make -j
    make install
    \end{lstlisting}
  \item If required, install the netcdf4-python module. 
    Change to the root-directory of the esdm-netcdf repository and install the patched netcdf-python module.
    \begin{lstlisting}
    cd dev
    git clone https://github.com/Unidata/netcdf4-python.git
    cd netcdf-python
    patch -s -p1 < ../v1.patch
    python3 setup.py install --user
    \end{lstlisting}
\end{enumerate}

\subsection{Docker}
For the development of ESDM the directory \lstinline|./dev/docker| contains all requirements to quickly set up a development environment using docker.
The Dockerfiles contain ESDM installation instructions for different distributions.
For easy building Dockerfiles for different plattforms are provided in different flavours:

\begin{itemize}
  \item CentOS/Fedora/RHEL like systems
  \item Ubuntu/Debian like systems
\end{itemize}

\subsubsection{Setup}

Assuming the docker service is running you can build the docker images as follows:

\begin{lstlisting}
cd ./dev/docker
cd <choose ubuntu/fedora/..>
sudo docker build -t esdm .
\end{lstlisting}

After docker is done building an image, you should see the output of the ESDM test suite, which verifies that the development environment is set up correctly. 
The output should look similar to the following output:

\begin{lstlisting}
Running tests...
Test project /data/esdm/build
Start 1: metadata
1/2 Test #1: metadata .........................   Passed    0.00 sec
Start 2: readwrite
2/2 Test #2: readwrite ........................   Passed    0.00 sec

100% tests passed, 0 tests failed out of 2

Total Test time (real) =   0.01 sec
\end{lstlisting}


\subsubsection{Usage}
Running the esdm docker container will run the test suite:

\begin{lstlisting}
sudo docker run esdm  # should display test output
\end{lstlisting}

You can also explore the development environment interactively:

\begin{lstlisting}
sudo docker run -it esdm bash
\end{lstlisting}

\begin{lstlisting}
docker run dev/docker/ubuntu-whole-stack/Dockerfile
\end{lstlisting}


%\subsubsection{NetCDF with Support for ESDM VOL Plugin}%
%\label{netcdf-with-support-for-esdm-vol-plugin}

%Assuming all prerequisites have been installed and tested a patched version of NetCDF to enable/disable ESDM VOL Plugin support can be build as follows.

%\subsection{Ensure environment is aware of dependencies installed using spack and dev-env}%
%\label{ensure-environment-is-aware-of-dependencies-installed-using-spack-and-dev-env-1}

%\subsection{Download NetCDF source}%
%\label{download-netcdf-source}

%version=4.5.0

%\begin{lstlisting}
%wget ftp://ftp.unidata.ucar.edu/pub/netcdf/netcdf-$version.tar.gz
%cd netcdf-$version
%\end{lstlisting}

%\subsection{Apply patches to allow enabling/disabling plugin}%
%\label{apply-patches-to-allow-enablingdisabling-plugin}

%\begin{lstlisting}
%patch -b --verbose $PWD/libsrc4/nc4file.c $ESDM_REPO_ROOT/dev/netcdf4-libsrc4-nc4file-c.patch
%patch -b --verbose $PWD/include/netcdf.h $ESDM_REPO_ROOT/dev/netcdf4-include-netcdf-h.patch
%\end{lstlisting}

%\subsection{Configure and build}%
%\label{configure-and-build}

%\begin{lstlisting}
%export CC=mpicc
%mkdir build && cd build
%cmake \
%  -DCMAKE_PREFIX_PATH=$prefix \
%  -DCMAKE_INSTALL_PREFIX:PATH=$prefix \
%  -DCMAKE_C_FLAGS=-L$prefix/lib/ \
%  -DENABLE_PARALLEL4=ON ..
%make -j
%make test
%make install
%\end{lstlisting}


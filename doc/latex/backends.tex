\section{Supported data backends}
ESDM uses at least one storage backends but can use multiple to store the data and exactly one metadata backend.

\subsection{DDN Web Object Scaler (WOS)}
DDN's Web Object Scaler or WOS is a distributed object storage system designed for extremely large data. 
It is available as a complete hardware/software solution or can be deployed on third-party hardware.
Although, the nodes can be geographically distributed, they present a shared pool for object storage. 
The data can be accessed over HTTP/REST, WebDAV, Swift, and S3 interfaces.
Additionally, the pools can be mounted via NFS and CIFS/SMB. 

\subsection{DDN Infinite Memory Engine (IME)}
IME is software with a server and a client component. 
Rather than issuing I/O to a parallel file system client, the IME client intercepts the I/O fragments and issues these to the IME server layer which manages the NVM media and stores and protects the data.
Prior to synchronizing the data to the backing file system, IME coalesces and aligns the I/O optimally for the file system. 
The read case works in the reverse: file data is ingested into the cache efficiently in parallel across the
IME server layer and will satisfy reads from here in fragments according to the read request. 
IME manages at-scale flash to eliminate file system bottlenecks and the burden of creating and maintaining application-specific optimizations. 
It delivers:

\begin{itemize}
  \item New levels of I/O performance for predictable job completion in even the most demanding and complex high-performance environments.
  \item Performance scaling independent of storage capacity for system designs with order of magnitude reductions in hardware.
  \item Application transparency that eliminates the need to create and maintain application-specific optimizations.
\end{itemize}

\subsection{Seagate Motr (MOTR)}
Motr is an object storage system developed by Seagate to overcome typical limitations of traditional storage systems.
In contrast to similar I/O storage system (e.g. Ceph and DAOS) Motr assesses raw block devices directly.

The design of Motr supports raw data and metadata.
To achieve that it offers two types of objects:
(1) A common object which is an array of fixed-size of blocks. 
Data can be read from and written to these objects. 
(2) An index for a key-value store. 
Key-value records can be put to and get from an index. 
At the moment, Motr provides a C interface.

\subsection{Portable Operating System Interface (POSIX)}
The Portable Operating System Interface (POSIX) is a family of standards specified by the IEEE Computer Society for maintaining compatibility between operating systems.
POSIX defines the application programming interface (API), along with command line shells and utility interfaces, for software compatibility with variants of Unix and other operating systems.

\subsection{Kove Direct System Architecture (KDSA)}
Kove External Memory~\cite{10.1007/978-3-319-67630-2_48} allows the CPU access to unlimited memory right below the CPU cache. 
Any data set can live in memory, close to the core, reducing processing time.
The KDSA API is a low-level API that allows to access data on external memory by utilizing RDMA. 
Data can be transferred synchronously or asynchronously, additionally, memory can be pre-registered for use with the Infiniband HCA. 
Since registration of memory is time consuming, for unregistered memory regions the system may either use an internal (pre-registered) buffer and copy the user’s data to the buffer, or for larger accesses it registers the memory, performs an RDMA data transfer and then unregisters the memory again.

\subsection{Amazon Simple Storage Service (S3)} 
Amazon Simple Storage Service (S3) is a scalable, web-based, high-speed cloud storage service with a simple-to-use API.
The service allows saving and archiving of data reliably in the Amazon Web Services (AWS) of all sizes.
It is suitable for use cases, like backup, archives, big-data, IoT devices, websites and much more.

\section{Supported data backends}
\subsection{DDN Web Object Scaler (WOS)}
%DDN (Data Direct Network) WOS (Web object scaler)4 represents an object storage solution able to manage files as “objects”.
%It offers a simple and effective way to manage data stored in the cloud through an ease administration interface and IP based direct connection to the nodes. 
%WOS architecture is natively geographically agnostic, and this represents one of the main features of the product: nodes can be deployed anywhere, and the access to data which they host is guaranteed by Internet Protocol (IP) connectivity. 
%In this sense, all the nodes which form the cloud work together to create an aggregated pool of storage space.

Web Object Scaler or WOS is a distributed system with objects managed on nodes that are servers running the custom software with attached disk storage. 
The nodes can be geographically dispersed but present a common pool for object storage. 
A custom API over HTTP/REST is supported along with WebDAV, Swift, and S3. 
A file interface for NFS and CIFS/SMB is supported as well with a WOS file system gateway. 
WOS is delivered as a complete hardware and software system or WOS software for integration with other hardware.
As the name implies, WOS was designed for the high capacity, large number of data elements required for web or cloud environments. 
Scaling to petabytes of unstructured data with geographic dispersion is the typical usage for WOS. 
Nodes are self-contained systems with CPU, memory, and storage. 
Configured in a federation with Ethernet interfaces, nodes can have different capabilities and performance. 
Each node is active in the configuration and objects are dispersed in the global namespace according to algorithms in the WOS policy engine to provide availability and durability.
Distributed performance is enabled by a feature called the WOS-Library that runs on application servers accessing the WOS cloud as a client. 
The WOS-Library uses a callback style API to WOS and maintains a map of the topology. 
WOS-Library also makes routing decisions about where to store data as part of load balancing. 
Object ID lookup in the topology map is fast because the map is kept in memory.

\subsection{Seagate Motr (MOTR)}
Motr is an Exascale ready Object Store system developed by Seagate and built from the ground up to remove the performance limitations typically found in other designs. 
Unlike similar storage systems (e.g. Ceph and DAOS) Motr does not rely on any other file system or raid software to work. 
Instead, Motr can directly access raw block storage devices and provide consistency, durability and availability of data through dedicated core components.
Motr offers two types of objects: (1) A common object which is an array of fixed-size of blocks. 
Data can be read from and written to these objects. (2) An index for a key-value store. 
Key-value records can be put to and get from an index. So Motr can be used to store raw data, as well as metadata.
Motr provides C language interfaces, i.e. Clovis, to applications. 
ESD middleware will use Clovis and link with Clovis to manage and access Motr storage cluster.

\subsection{DDN Infinite Memory Engine (IME)}
IME is software with a server and a client component. 
Rather than issuing I/O to a parallel file system client, the IME client intercepts the I/O fragments and issues these to the IME server layer which manages the NVM media and stores and protects the data.
Prior to synchronizing the data to the backing file system, IME coalesces and aligns the I/O optimally for the file system. 
The read case works in the reverse: file data is ingested into the cache efficiently in parallel across the
IME server layer and will satisfy reads from here in fragments according to the read request. 
IME manages at-scale flash to eliminate file system bottlenecks and the burden of creating and maintaining application-specific optimizations. 
It delivers:

\begin{itemize}
  \item New levels of I/O performance for predictable job completion in even the most demanding and complex high-performance environments.
  \item Performance scaling independent of storage capacity for system designs with order of magnitude reductions in hardware.
  \item Application transparency that eliminates the need to create and maintain application-specific optimizations.
\end{itemize}

\subsection{Portable Operating System Interface (POSIX)}
The Portable Operating System Interface (POSIX) is a family of standards specified by the IEEE Computer Society for maintaining compatibility between operating systems.
POSIX defines the application programming interface (API), along with command line shells and utility interfaces, for software compatibility with variants of Unix and other operating systems.

\subsection{Kove Direct System Architecture (KDSA)}
Kove External Memory~\cite{10.1007/978-3-319-67630-2_48} allows the CPU access to unlimited memory right below the CPU cache. 
Any data set can live in memory, close to the core, reducing processing time.
The KDSA API is a low-level API that allows to access data on external memory by utilizing RDMA. 
Data can be transferred synchronously or asynchronously, additionally, memory can be pre-registered for use with the Infiniband HCA. 
Since registration of memory is time consuming, for unregistered memory regions the system may either use an internal (pre-registered) buffer and copy the user’s data to the buffer, or for larger accesses it registers the memory, performs an RDMA data transfer and then unregisters the memory again.

\subsection{Amazon Simple Storage Service (S3)} 
Amazon Simple Storage Service (S3) is a scalable, web-based, high-speed cloud storage service with a simple-to-use API.
The service allows saving and archiving of data reliably in the Amazon Web Services (AWS) of all sizes.
It is suitable for use cases, like backup, archives, big-data, IoT devices, websites and much more.
